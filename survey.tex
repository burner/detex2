\documentclass[conference]{IEEEtran}

\ifCLASSINFOpdf
\else
\fi

%\usepackage{biblatex}
%\addbibresource{biblio.bib}
\usepackage[
    backend=biber,
    style=authoryear-icomp,
    sortlocale=en_US,
    sorting=none,
    natbib=true,
    url=false, 
    doi=true,
    style=numeric,
    eprint=false
	%,backref
]{biblatex}
\addbibresource{biblio.bib}

\input{preamble}

\begin{document}

\title{A Survey of Static, Syntactic, Structured and Unstructured Quorum Protocols}

\author{\IEEEauthorblockN{Robert Schadek and Oliver Theel}
\IEEEauthorblockA{Carl von Ossietzky University of Oldenburg\\
Department of Computer Science\\
\{robert.schadek,theel\}@informatik.uni-oldenburg.de}}

\maketitle
\thispagestyle{plain}
\pagestyle{plain}

\begin{abstract}
Quorums are a powerful concept that can be used to implement highly available
services.
Quorum protocols use replicas of data to increase service availability.
These quorum protocols can exbibit a variety of different properties.
This paper will systematically introduce these properties and then use them
to analyze existing quorum protocol that bear historical relevance.
Additionally an algorithm will be presented that makes it possible to fairly
compare different kinds of quorum protocols.
\end{abstract}

\begin{keywords}
Structured Quorum-Systems, Replication, One-Copy-Serializability, Concave Hulls
\end{keywords}

\Section{Introduction}
Data replication has a long history in computer science.
Highly available access to a data object is a core problem of computer
science.
Relying on a single replica of a data object greatly limits the availability
of said data.
Therefore, the next logical step is to increase the number of replicas storing
the data object.
This increases the availability of the data object as it can be accessed from
different replicas.
Replicating the data object also introduces the need for synchronizing.
Imagining a data object that is replicated on five replicas as shown in 
\Rl{fig:fivereplicas}.
\begin{figure}
\centering
	\includestandalone{5replica}
\caption{Five replicas storing the same data object}
\label{fig:fivereplicas}
\end{figure}
When the data object located on replica \(0\) is overwritten with a new value,
reading the data object from replica \(4\) will not yield the new value.
Obviously, that is not the intend behavior.
An additional problem is that two concurrent write operations can be executed
on different replicas on the same data object at the same time.
For example, value \(0\) is written to the data object on replica \(2\) and
value \(1\) is written to the data object on replica \(3\).
This begs the question which value is the correct value.
Both examples show that simple creating multiple replicas of an data
object will not necessarily lead to the wanted results.
Usually, the wanted result is that all operations behave as they
would do on a non-replicated data object but with greater availability.
More formal, this \emph{non-replicated behaviour} is known as the \g{ocs}
property \cite{BHG87:ccr}.
It basically describes that replicated data behaves as non replicated data
would.
Meaning, every read operation reads the last written data, only one write
operation can be executed at any given time, and read and write operation can
not be executed in parallel.

Algorithm for \emph{mutual exclusion} are closely related to data replication
protocols that adhere the \g{ocs} property.
In mutual exclusion, any process that executes an mutual exclusion algorithm
either is allowed to enter the critical section or not.
Comparing this with the intersection properties of read and write operations,
and write and write operation intersection of data replication protocols
presents an obvious similarity.
Entering a critical section mutually exclusiv to read, or to modify data, is
the virtual the same as using read and write operation that create mutual
exclusion for the duration of its activity.

In this paper we will focus on \gp{qp} that implement this \g{ocs} property.
Additionally, the evolution and terminology of quorum protocols will
discussed, as well as different quorum protocols and there characteristics.
In \Rl{sec:systemmodel} we will introduce the terminology, and the system
model used.
Common \gp{qp} and their historical context will be presented in presented
in \Rl{historicaldevelopmentandcomparision}.
A generic approach to compare the performance of \gp{qp} is given in
\Rl{sec:availabilitycomparison}.
\Rl{sec:conclusion} is concluding this paper.

\Section{System Model}
To thoroughly discuss different quorum protocols, and the characteristics we
first need define coherent model.
A graph-structure \(GS^{D} = (V,E)\) is defined as a two-tuple with a set of vertices \(V\) and
a set of edges \(E\).
The \(D\) donates the dimension of the graph-structure.
When \(D = 2\) it is omitted.
Edges connect vertices.
\(V = V(GS)\) donates the set of vertices of a \(GS\).
\(E = E(GS)\) donates the set of edges of a \(GS\).
A vertex \(v \in V\) is defined as a tuple \(v = (i,c_1, \dots, c_{D})\), where 
\(i \in \mathbb{N}\) is the id of the vertex and \(c_1, \dots, c_{D}\) are
the coordinates of the node in the corresponding dimensions, where the
coordinates values are natural numbers.
The shorthand notation \(v_i\) donates a vertex with id \(i\).
\(|V(GS)|\) represents the number of vertices in a \(GS\).

An \emph{edge} \(e_{i,j} \in E\) is defined as \(e_{i,j} := (v_i, v_j)\), where
\(v_i, v_j \in V\).

A path \(\langle v_0, v_1, \dots, v_n \rangle\) exists in \(GS^D\) iff:
\begin{align}
\forall i &, 0 \le i \le n : \exists\,v_i \in V \text{ and }\\
\forall i &, 0 \le i < n : \exists e = (v_i,v_{i+1}) \in E 
\end{align}
If a path exists, the two nodes \(v_0\) and \(v_n\) are called connected.
\(\mathbb{V}(\langle v_0, v_1, \dots, v_n \rangle )\) denotes the set of vertices of
a path such that:
\begin{align}
	\mathbb{V}(\langle v_0, v_1, \dots, v_n \rangle ) =& \{v_0, v_1, \dots, v_n\}.
\end{align}
\(\mathbb{E}(\langle v_0, v_1, \dots, v_n \rangle )\) denotes the set of the
edges of a path such that:
\begin{align}
	\mathbb{E}(\langle v_0, v_1, \dots, v_n \rangle ) =& \{e_{0,1}\dots, e_{n-1,n}\}.
\end{align}
The function \(GS(V',E')\) creates a new graph structure for which:
\begin{align}
	V(GS(V',E')) &= V'\\
	\forall e_{i,j} \in E(GS(V',E')) &: e_i \in V' \land e_j \in V'
\end{align}
is true.
\Paragraph{Data Object}
A \emph{data object} \(d\) is a single value, its actual value, or data type is of
no concern.
It possible to read, and write a data object.
\(D\) is the set of all \(d\) in a system.

\Paragraph{Replica}
A replica is a copy of a data object.

\Paragraph{Node}
A \emph{node} is a logical aggregation between one vertex and zero or more
replicas.

\paragraph{\Gp{pnt} and \Gp{lnt}}
A \g{pnt} is the actual network topology that connects the nodes of the quorum
protocol and transfers their messages.
The topology can have arbitrary shape and is described by a graph-structure.

A \g{lnt} is a network topology that is put on top of a \g{pnt}.
They are also described by a \(GS^{D}\).
As the name indicates, these topologies are only logical and must not be
homomorph to the underlying \g{pnt}.
If the \g{lnt} and the \g{pnt} are not homomorph the message complexity given
for a \g{lnt} can therefore only be an approximation for the actual message
complexity when using the \g{pnt}.

\Paragraph{One-Copy-Serializability}
\begin{quote}
the result of any execution is the same as if the operations of all the
processors were executed in some sequential order, and the operations of
each individual processor appear in this sequence in the order specified by
its program. \cite{Lam79:hmm}
\end{quote}
For a formal definition of \g{ocs} please refer to
\cite{BHG87:ccr,Lam79:hmm}.
The idea behind \g{ocs} is that read and write operations, and multiple write
operations are executed mutually exclusive.

\Paragraph{Coterie}
The concept of coteries was first presented in \cite{Lam78:ird} and later
reintroduced by \cite{GB85:hav}.
Let \(R\) be a set of elements.
A coterie \(C\) is a set of sets of elements for which the following is true:
\begin{align}
	\forall G \in C &: G \ne \emptyset \wedge G \subset R \\
	\forall G,H \in C &: G \ne H \wedge G \cap H \ne \emptyset \\
	\forall G,H \in C &: G \ne H \wedge G \not\subset H
\end{align}

\Paragraph{Fully Distributed}
\Gp{qp} are considered to be fully distributed if no node has a special role or
is more important as any other node.
Fully distributed \G{qp} are usually highly available as they are not depended
on a availability a specific node.
Single copy systems are the opposite of fully distributed system, as the
complete system is depended on a single node.

\Paragraph{Access Sites}
The access sites describe the place from where an operation is started.
An operation can either be started from inside the \(GS\) used by the \g{qp}
or from outside the \g{qp}.

\Paragraph{Operation Primitives}
Most \gp{qp} provide a read and a write operation.
Some protocols provide a specialized update operation instead of the write
operation.
These update operation read, modify the read data, and write the modified data
in an uninterruptible fashion.
The modification of the data is usually done by a user supplied function. 

\Paragraph{Operation Availability}
For the availability calculation it is assumed that all nodes exbibit the same
level of availability, and that node failure is independent from each other.
Furthermore, unless stated otherwise, it is assumed that each node holds one
replica.
The availability \(\mathbb{A}\) of an operation for a given \g{qp} under a given
node availability \(p\) is usually, uniquely calculated for that \g{qp}.
Some \g{qp} have a closed formula for the availability calculation.
If \(Q\) is a set of all quorums for an operation of a \g{qp}, and these
quorums upholds the coterie property, the following formula can be used to
calculate the availability of the operation.
Where \(N\) is the number of nodes of the \g{qp} which are available with a
probability \(p\) and \(|q|\) donates the number of nodes of a quorum \(q \in
Q\).
\begin{align}
	\mathbb{A}(p,Q,N) = \sum_{q \in Q} p^{|q|} * (1-|q|)^{(N-|q|)} \label{eq:avail}
\end{align}
\Rl{eq:avail}
This formula only gives information about the operation availability or \g{qp}
where the \g{pnt} plays no role or is not considered.

\Paragraph{Operation Costs}
The cost of an operation can be measured in many different ways.
Some measurements are based on the number of message required to send to
execute the operation.
The more common approach is based on counting the number of nodes involved in
the operation.
\Rl{eq:cost} presents this cost measurement that allows to compare different
\g{qp} solely based on the quorums these \g{qp} construct.
It is again assumed that \(Q\) is a coterie.
The variables \(q,p\) have the same meaning as in the previous Paragraph.
\begin{align}
	\frac{\sum_{q \in Q} p^{|q|} * |q|}{|Q|} \label{eq:cost}
\end{align}
The idea behind \Rl{eq:cost} is to weight the cost of each quorum by the
probability of its availability, and then average the costs of all quorums.
By costs of an quorum we mean the number of nodes required to construct said
quorum.

\Paragraph{Discussion}
The topics Access Sites, \g{lnt}, \g{pnt} and fully distributed are tightly
coupled.
If for instance we consider the graph-structure of \Rl{fig:tree} and assume
that all access must go through node \(0\) the resulting protocol will not be
fully distributed, as node \(0\) must always be available for any operation to
executed.
If we assume that the nodes in \Rl{fig:fivereplicas} are completely connected
and any operation can be started by any of the nodes the resulting \g{qp} is
fully distributed.
So far only the \g{lnt} have been considered.
Usually, the \g{lnt} used by \gp{qp} are not homomorph to \g{pnt} found in the
real world.
Therefore, in order to give a definitive answer about the fully distributed
property the \g{pnt} has to be considered as well.

\Section{Historical Development and Comparison} \label{historicaldevelopmentandcomparision}
Quorum Protocols where introduced around the time of the Arpanet.
In the following different \gp{qp} will be presented and there
advantages and disadvantages will be discussed.

The most obvious approach to assure \g{ocs} is the \g{rowa} Protocol.
As the name implies read operation read one node and write operations write
all nodes.
The write availability \(a_w(p)\) of the \g{rowa} Protocol, where \(p\) is node
availability of all \(N\) nodes in the system, is:
\begin{align}
	p^{N}
\end{align}
and the read availability \(a_r(p)\) is
\begin{align}
	1 - (1 - p)^{N}.
\end{align}
The read availability of the \g{rowa} Protocol is obviously very good, but the write
operation is hardly availability for a growing number of nodes.

A modification of the \g{rowa} Protocol is the \g{rowaa} Protocol.
Here the write operation only has to write the available nodes.
The \g{rowaa} Protocol does not provide \g{ocs}.
The write availability of the \g{rowaa} Protocol is \(1\), as writing no nodes
is always possible.

In \cite{Johnson:1975:MDD:RFC0677} \citeauthor{Johnson:1975:MDD:RFC0677}
discussed the maintance of a duplicated database.
The inital concept was extended in \citetitle{Tho79:mca} by
\citeauthor{Tho79:mca}.
Compared to later publications about data replication the presented protocol
is very pratical oriented.
The publication discusses the implementation of a replicated \g{dbms} in
detail.
For example, the authors discusses whether to use daisy chains or broadcasts
for the communication between the \gp{dbms}.
The replicated database is replicated among \(N\) \gp{dbms}.
\Gp{ap} \emph{query} a single \g{dbms} for data.
In addition to the data the \g{dbms} also returns the timestamp of the data's
last modification.
The \g{ap} can modifies the data according to its business logic.
Then the \g{ap} uses the \emph{update} procedure to write back the modified
data to a single \g{dbms}.
The \g{dbms} then requires \(\lceil \frac{N}{2} \rceil\) messages to achieve
consensus between \(\lceil \frac{N}{2 }\rceil\) \gp{dbms} whether or not to
accept the update.
Then an additional \(N-1\) messages are required to inform all \gp{dbms} about
the acceptance of the update and its corresponding data.
Finally, the \g{ap} is informed about the acceptance of its update.
\G{mcs} does not provide \g{ocs}, as reading is permitted while a writing
operation is in progress.
Consider the following example.
A write operation has acquired consensus by the majority of the \gp{dbms}, but
not all \gp{dbms} have been informed about the modified data.
A read operation reads the data from an uninformed \g{dbms}.
After the read operation is finished the \g{dbms} is informed that the data
was changed even before the read operation was started.

Historically, \g{mcs} has been used differently.
Usually, no distinction is made between \gp{ap} and \gp{dbms}, they are simply
combined into nodes.
Additionally, read operations do not read from a single node, but also from
the majority of the nodes.
Thereby, creating the intersection property between every \g{qr} and every
\g{qw}.
This version provides \g{ocs}.
The write availability \(a_w(p)\) of this variant, where \(p\) is node
availability of all nodes \(N\) in the system, is:
\begin{align}
	p^{\lceil \frac{N}{2} \rceil}
\end{align}
and the read availability \(a_r(p)\) is
\begin{align}
	p^{\frac{N}{2}}.
\end{align}

\citetitle{Gif79:wvr} was presented in \cite{Gif79:wvr}.
In this protocol every node has a number of votes assigned to it.
A read quorum has to collect \(r\) read votes and a write quorum has to select
\(w\) votes replicas.
Additionally, \(r + w\) must be greater than the total number of votes \(v\)
in the system.
This protocol does enforce that read and write operation are executed mutual
exclusive, but is does not enforce that two write operations are mutually
exclusive.
\begin{theorem}[Weighted Voting Empty Write Quorum Intersection]
\label{thm:weightedemptyintersect}
	\(\exists (qw_1,qw_2) : qw_1 \cap qw_2 = \emptyset\)
\end{theorem}
\begin{proof}[Proof of \Rl{thm:weightedemptyintersect}]
	Assume \(N = 5\) replicas \(c_0, \dots, c_4\) each with \(1\) vote.
	\(r = 4\) and \(w = 2\).
	\(r + w > N\).
	\((c_0,c_1) \cap (c_2,c_3) = \emptyset\)
\end{proof}
This problem is usually solved by adding an additional requirement to the
protocol.
\begin{align}
	\forall w_1, w_2 : w_1 + w_2 >= N
\end{align}
The modified version of the \g{wvp} is very similar to \g{mcs}, but it allows
to favor nodes with higher availability and it allows to favor read operations
over write operations or vice versa.
Actually, consider a system where \(\lceil \frac{N}{2} \rceil\) nodes are
assigned one write vote and \(\lfloor \frac{N}{2} \rfloor\) nodes are one read
vote the \g{wvp} behaves equal to the \g{mcs}.
Actually, the modified \g{wvp} is a superset of the modified \g{mcs}.
No closing formula for the read and write availability of the \g{wvp} can be
given, as the vote assignment can be non-uniform across the nodes
\footnote{Compare \Rl{eq:avail}.}
Simplifying the vote assignment to one vote per node allows for such a
formula for a limited number of read- write-vote ratios.
Limited, because if the number of write votes is less than half of the votes
\(v\) the resulting system will no longer be \g{ocs} as write quorums must not
intersect anymore.
The write availability \(a_w(p)\) of this \g{wvp}, where \(p\) is the
node availability of all nodes \(N\) in the system, and \(r \in [0.5,1.0)\)
is the ratio between read and write votes, is:
\begin{align}
	p^{\lceil N \times r \rceil}
\end{align}
and the read availability \(a_r(p)\) is
\begin{align}
	p^{\lceil N \times (1 - r) \rceil}
\end{align}
Interestingly for \(\lim_{r\to1.0}\) the modified \g{wvp} is equal to the
\g{rowa} Protocol.

So far all presented \gp{qp} require no \g{lnt} nor \g{pnt}, implicitly though
the presented protocols require that there nodes are part of a complete graph.
The requirement of a complete graph as a quasi \g{lnt} becomes especially
problematic as shown in \Rl{sec:availabilitycomparison}.

The \(\sqrt{N}\) Mutual Exclusion Algorithm was presented in \cite{Mae85:ame}.
As discussed earlier, algorithm for mutual exclusion can often be used for data
replication purposes.
It is called \(\sqrt{N}\) Mutual Exclusion Algorithm as it requires to lock
\(\sqrt{N}\) nodes to obtain mutual exclusion.
The idea behind the algorithm is based on finite projection planes
\cite{AARS:ProjectionPlane}.
If there is a prime \(k\) such that:
\begin{align}
	N &= k^2 + k + 1 \label{eq:nsqrtk}
\end{align}
there exists a set called \(D\) of \(N\) paths for which
\begin{align}
	\forall a,b &\in D : |\mathbb{V}(a)| = |\mathbb{V}(b)| \label{eq:nsqrtequal}\\
	\forall a,b &\in D : \mathbb{V}(a) \cap \mathbb{V}{(b)} \ne \emptyset \label{eq:nsqrt}
\end{align}
is true.
Each path can be used as a read or write quorum.
\Rl{fig:nsqrt} shows an example for the \(\sqrt{N}\) Algorithm with seven
nodes.
The read and write quorums consists of all the nodes on one of the straight
paths or the circle.
For all \(N\) such that there is no \(k\) for which \Rl{eq:nsqrtk} holds,
degenerated sets can be created.
Degenerated in the sense that some sets hold more elements than other.
In other works \Rl{eq:nsqrtequal} is no longer true.
Approximating rounding errors the average set size in \(D\) for any number of
\(N\) will be \(\approx \sqrt{N}\) hence the name of the algorithm.
The graph in \Rl{fig:nsqrt} is also called a \emph{fano plane} or visual
representation of a \emph{Steiner System} \cite{assmus93steiner,Manivel06fano}.

The \citetitle{AA90:tqp} uses a logical tree structure to optimize the
availability of the write operation of the \g{rowa} Protocol \cite{AA90:tqp}.
The nodes used by the protocol are arranged in a tree as shown in
\Rl{fig:tree}.
\begin{figure}
	\includestandalone{tree}	
	\caption{Logical tree structure used by the \emph{Tree Quorum Protocol} and
	the \emph{Generalized Tree Quorum Protocol}.}
\label{fig:tree}
\end{figure}
The \g{tqp} requires that the write operation writes a \g{qw} that consists of
the root, a majority of its children, and a majority of their children, and so
forth.
The read operation has to read the majority of all nodes at a single level of
the tree.
Considering the tree in \Rl{fig:tree} a valid \g{qw} would be
\(\{0,1,3,4,9,10\}\).
At least for the write operation the protocol is not fully distributed, as it
always requires the root node to be present.
Again considering the tree in \Rl{fig:tree} a valid \g{qr} would be
\(\{1,2\}\).
The \g{tqp} had the premise to improve the \g{rowa} Protocol.
It achieves this goal but only for a limited amount of nodes.
As it is always required that the root node is part of a \g{qw}, the
availability of this node is the limiting factor for the write availability.
Therefore, for a system with \(\lim_{|N| \to \infty}\) the write
availability will eventually be \(0\) as long as the individual node
availability \(p \ne 1\).
As the concept of data replication was introduced because the node
availability was \(< 1\), this is trivially true.

The \g{tqp} was refined into the \citetitle{AA92:gtqp} \cite{AA92:gtqp}.
The difference in comparison to the \g{tqp} is the modified construction of
\gp{qr}.
Instead of requiring the majority of a single level, a \g{qr} in the
\g{gtqp} is required to consists of the majority of a level where an element
of a level can also be replaced by the majority of its children on the next
level.
A valid \g{qr} for the \g{gtqp} in the tree in \Rl{fig:tree} is \(\{1,6,8\}\).
The requirements for the \gp{qw} are unchanged in comparison to the \g{tqp}.

\emph{Multidimentional Voting} is an generalisation of the voting protocol.
It was introduced in \cite{AA91:mdv,CAA90:mdv}.
The idea behind Multidimentional Voting is that every node has a fixed number
of votes not on a single dimension but on multiple dimensions. 
In order to execute an operation a quorum has to be found for a number of
\(l\) dimension.
\begin{table}
\begin{tabular}{l l l l l l l l }
	& \rotatebox{90}{Dimension 1} & 
	\rotatebox{90}{Dimension 2} & 
	\rotatebox{90}{Dimension 3} & 
	\rotatebox{90}{Dimension 4} & & \\
	\textbf{Node 1} & \vrule 2 & 0 & 2 & 2 \vrule&  2 0 2 2 & \(\leftarrow\)selected Node \\
	        Node 2  & \vrule 3 & 1 & 0 & 0 \vrule&          & \\
	\textbf{Node 3} & \vrule 0 & 1 & 0 & 2 \vrule&  0 1 0 2 & \(\leftarrow\)selected Node \\
	\textbf{Node 4} & \vrule 1 & 0 & 1 & 1 \vrule&  1 0 1 1 & \(\leftarrow\)selected Node \\
	        Node 5  & \vrule 1 & 0 & 1 & 1 \vrule&          & \\
	        Node 6  & \vrule 0 & 1 & 2 & 0 \vrule&          & \\\cmidrule{6-6}
					& \multicolumn{4}{c}{Vote Vector} &  3 1 3 5 & sum of votes per dimension \\
		   	        &          &   &   &         &  5 3 5 5 & quorum requirements \\\cmidrule{6-6}
		   	        &          &   &   &         &  0 0 0 1 & vector of valid dimension
\end{tabular}
	\caption{Example of Multidimentioanl Voting as presented in \cite{AA91:mdv,CAA90:mdv}.}
	\label{tab:multidimvoting}
\end{table}
Considering the example in \Rl{tab:multidimvoting} and a Multidimvoting Voting
Protocol \(MD(l,k)\) with \(\mathit{l} = 1\) and \(k = 4\) we will shortly
discuss its operation.
\(\mathit{l}\) defines the number of dimensions where a valid quorum must be
found and \(k\) defines the number of dimensions of the voting vector.
The example given finds the fours dimension as a suitable voting dimension, and
as \(\mathit{i} = 1\) that is enough to execute the operation.
The challenge with the Multidimentional Voting Protocol is the assignment of
the votes and the quorum requirements.
The votes have to be assigned in such a way that all valid quorum
have an intersection with all other valid quorums over all dimensions.

The \emph{\G{mlq}} was introduced in \cite{AA91:mdv,FKT91:dml}.
A similar approach was presented by \citeauthor{Kum90:pah} in
\cite{Kum90:pah,Kum91:hqc}.
\begin{figure}
	\includestandalone[width=0.48\textwidth]{multilevel}	
	\caption{An example for a Multi-Level Replication Management Protocol.}
	\label{fig:multilevel}
\end{figure}
The idea behind the \g{mlq} is to aggregate existing nodes into
logical group.
And then use these logical to execute an arbitrary \g{qp}.
Furthermore, these logical groups are then grouped again on the next logical
level and so forth until the root level is reached.
Obviously, the \g{mlq} works on \gp{lnt}.
As an example consider \Rl{fig:multilevel}.
Nine physical nodes are group into three \g{mcs} strategies, which are in turn
grouped into a \g{rowa} strategy.
When an operation is executed on this \g{mlq} the operations starts at the
level 0 \g{rowa} strategy.
A read operation has to read one logical node of level 1.
Simply because \g{rowa} requires to read one node.
Every logical node at level 1 is implemented by the \g{mcs} strategy.
Therefore, one logical node at level 1 has to read the majority of its
children at level 2.
A write operation again has to start at level 0.
As level 0 is implemented by the \g{rowa} Protocol all logical nodes of level
1 have to be written.
As the logical nodes of level 1 are implemented by the \g{mcs} strategy they
only have to write the majority of there children at level 2.
Instead of writing all nine node, the \g{mlq} only requires to write six
nodes, reading on the other hand now requires two nodes instead of one.
The \g{mlq} approach is very flexible as it allows to assign different data
replication protocol to every logical node.
Additionally, the number of children associated very every logical node can
vary.
As such, the costs and the availability of the operations have to be computed
individually for each \g{mlq} instantiation.
In \cite{Kum91:hqc} it is shown that the read and write costs for some of the
hierarchical structures can be as good as \(N^{0.63}\).

The \g{gp} was presented in \cite{Kum90:pah} by \citeauthor{Kum90:pah}.
A grid used by the \g{gp} is shown in \Rl{fig:grid}.
In the general case the grid has to be considered a \g{lnt}, as regular grid
structures are rarely found in real-world networks.
\Gp{qr} are constructed by selecting one node of each column.
For a grid with \(R\) rows and \(C\) column the read availability \(a_r(p)\)
is:
\begin{align}
	(1 - (1 -p)^C)^R
\end{align}
The read costs for the \g{gp} is equal to the number of column.
As an example for a valid \g{qr} for the \g{gp} in \Rl{fig:grid} consider the
nodes \(\{4,9,2,15\}\).
\Gp{qw} are constructed by selecting one node of each column and one complete
column.
\begin{figure}
	\begin{subfigure}[t]{0.48\linewidth}
		\includestandalone[width=\linewidth]{grid}	
		\caption{Logical Grid structure used by the \emph{Grid Protocol}.}
		\label{fig:grid}
	\end{subfigure} ~
	\begin{subfigure}[t]{0.48\linewidth}
		\includestandalone[width=\linewidth]{lattice}	
		\caption{Logical triangular Lattice structure used by the \emph{Triangular Lattice Protocol}.}
		\label{fig:lattice}
	\end{subfigure}
	\caption{Grid based \g{lnt}.}
\end{figure}
A corresponding \g{qw} to the previous given \g{qr} for the Grid Protocol is
\(\{12,1,2,7,11,15\}\).
The nodes of the \g{qr} and the nodes of the \g{qw} of this example intersect
in node with id \(15\).
The write availability \(a_w(p)\) of the \g{gp} is:
\begin{align}
	(1 - (1 - p)^C)^R - (1 - p^C - (1 - p)^C)^R
\end{align}
The write costs for the \g{gp} is equal to \(C+R-1\).

The Grid Protocol was extended into the \emph{\g{tl}} \cite{WB92:tlp}.
The \g{tl} uses a \g{lnt} as shown in \Rl{fig:lattice}.
Considering the \g{lnt} used by the \g{gp} and the \g{lnt} used by the \g{tl}
the relation is obvious.
Every \g{qr} consists of a vertical or horizontal path through the \g{lnt}.
Every \g{qw} consists of a vertical and horizontal path through the \g{lnt}.
Such a protocol could also leverage \gp{lnt} as shown in \Rl{fig:grid}, the
additional edges forming the triangular lattice are an optimisation.
The diagonal connecting the nodes \({0,5,10,15}\) creates an minimal path that
crosses the \g{lnt} vertically and horizontally.
Therefore, as long as the \g{lnt} used by the \g{tl} is based on an \(M \times
M\) grid the optimal \g{qw} and \g{qr} will consists of \(M\) nodes.
As quorums for the \g{tl} are created by finding paths through a \g{lnt}, no
closed formula exists that calculates the read and write availability.
\Rl{fig:tlrea} shows the read and write availability of the \g{tl} by applying
the \Rl{eq:avail} on the \g{lnt} in \Rl{fig:lattice}.
\begin{figure}[b]
	\hspace{-10mm}%
	\includegraphics[width=1.2\linewidth]{tl_avail}
	\caption{\G{tl} Read and Write Availability with 16 Nodes.}
	\label{fig:tlrea}
\end{figure}
Both the \g{gp} and the \g{tl} face a similar problem.
The are only well defined if the number of nodes is \(N = M \times U\) where
\(M,U\) are the number of nodes per dimension.
As an example consider the \g{gp} and the \g{lnt} in \Rl{fig:grid} only
consisting of \(15\) nodes.
One row and column will never be complete, making the other nodes of that row
and column more important.
They are more important as now there is only a one in three chance to find a
available node in that column to construct a \g{qr}.
In all other columns the chance is one in four.
This change transforms the \g{gp} as well as the \g{tl} from a fully
distributed protocol to a not fully distributed protocol.

In \cite{The93:diss} an approach was presented to construct grid quorum
protocol to met certain read and write availabilities.

The \g{cw} was presented in \cite{PW94:cw:TR,PW97:acw}.
It uses a irregular grid like structure as shown in \Rl{fig:wall}.
\begin{figure}
	\begin{subfigure}[t]{0.48\linewidth}
		\includestandalone[width=\linewidth]{wall}	
		\caption{A \g{lnt} used by the Crumbling Wall Protocol.}
		\label{fig:irregularwall}
	\end{subfigure} ~
	\begin{subfigure}[t]{0.48\linewidth}
		\includestandalone[width=\linewidth]{logwall}	
		\caption{A logarithmic \g{lnt} used by the Crumbling Wall Protocol.}
		\label{fig:logwall}
	\end{subfigure}
	\caption{\Gp{lnt} used by the Crumbling Wall Protocol.}
	\label{fig:wall}
\end{figure}
The \g{cw} has one kind of quorum, it is constructed by creating the union of
all nodes from one row and one node from all rows below.
As an example for the intersection property of the \g{cw} consider the graph
in \Rl{fig:irregularwall} and the two sets of nodes \(\{11,12,14\}\),
\(\{4,5,6,7,12,13\}\).
It is not a fully distributed \g{qp} as the nodes of the lower rows are
required more often than the nodes of the upper rows.
In \cite{PW97:acw}, the authors of the Crumbling Walls Protocol
even show that a logarithmic wall as shown in \Rl{fig:logwall} is the only
highly available wall.
As such it is clear that the read and write availability is depended the wall
used.

\emph{The Wheel Quorum Consensus Protocol} was presented in
\cite{pa:2014:novel}.
The Wheel Quorum Consensus Protocol is a \g{rfwm} protocol.
The idea with that is that the amount of read operations dominates the amount
of write operations by an order of magnitude.
That means that read operations will require fewer nodes to create a valid
quorum than write operations.
As the protocol favors read operations over write operations, it aims to
decrease the read quorum size compared to protocols that do not favor a
specific operations.
\begin{figure}[ht]
	\begin{subfigure}[t]{0.48\linewidth}
		\includestandalone[width=0.98\textwidth]{wheel}
		\caption{A \g{lnt} used by the \g{wqp}.}
		\label{fig:wheel}
	\end{subfigure}
	\begin{subfigure}[t]{0.48\linewidth}
		\includestandalone[width=0.98\textwidth]{steiner}	
		\caption{A representation of the sets used by the \(\sqrt{N}\) Algorithm
			to reach mutual exclusion.}
		\label{fig:nsqrt}
	\end{subfigure}
	\caption{}
\end{figure}
\Rl{fig:wheel} shows the \g{lnt} used by the \g{wqp}.
The node in the middle is called the \emph{HUB}.
A \g{qr} will always only consist of the HUB node.
So in the case that the HUB node is available the read costs will be \(1\).
A \g{qw} consists of the majority of the nodes of the system and the HUB node.
As an example for the intersection property of \g{wqp} consider the
\g{qr} \(\{0\}\) and a \g{qw} \(\{0,1,2,3,4,5\}\) in the graph in
\Rl{fig:wheel}.
The \g{wqp} is obviously not fully distributed as all quorums require
the HUB node.
Thereby the availability of the read operation is equal to the availability of
the HUB node, and the availability of the write operation is always less than
the availability of the HUB node.
The HUB node is a single point of failure that undermines the idea of data
replication.
To correct this problem, the \g{wqp} elects a new HUB whenever its load
reaches a certain threshold or when the HUB is not availability.
As described in \cite{pa:2014:novel} a new HUB is elected by randomly
selecting two adjacency nodes and promoting the one with the most current
value to be the new HUB.
\begin{theorem}[The \g{wqp} is not \g{ocs}]
\label{thm:wheelprotocolisnotocs}
One property of \g{ocs} is that always the last written data is read.
Formally this can be expressed that their is always an intersection between
any \g{qw} and any \g{qr}.
\begin{align}
	\forall q_r, q_w : q_r \in QR \wedge q_w \in QW \wedge q_r \cap q_w \ne
	\emptyset
\end{align}
\end{theorem}
\begin{proof}[Proof of \Rl{thm:wheelprotocolisnotocs}]
Consider the wheel in \Rl{fig:wheel}.
\begin{enumerate}
	\item A \g{qw} writes the nodes \(\{0,1,2,3,4,5\}\).
	\item A new HUB gets elected from the two adjacency nodes \(7,8\).
	\item Node \(7\) is selected as new HUB.
	\item As required by the protocol the HUB is read.
\end{enumerate}
Obviously the read quorum does not intersect with the previously used write
quorum.
Therefore, the \g{ocs} property is not upheld by the \g{wqp}.
\end{proof}
Considering \Rl{thm:wheelprotocolisnotocs} the read operation costs of the
\g{wqp} is obviously not \(1\).
The selection of a viable new HUB requires at least the majority of all nodes
to produce an intersection with every possible \g{qw}.

In \cite{CJ98:tgp} and \cite{ChoW98} the authors presented a \g{qp} that works
on a \gp{lnt} with a triangular grid shape as shown in \Rl{fig:trigrid}.
\begin{figure}[ht]
\centering
	\includestandalone{trigrid}
	\caption{A \g{lnt} useable by the Triangular Grid Protocol}
\label{fig:trigrid}
\end{figure}
The protocol categorises nodes into three sets called \emph{inner nodes},
\emph{boundary nodes}, and \emph{corner nodes}.
There are three sets of boundary nodes the left boundary, the right boundary,
and the bottom boundary.
Corner nodes are nodes that are part of two boundary sets, for example node
number one in the graph shown in \Rl{fig:trigrid}.
Inner nodes are nodes that are not part of the boundary of the graph.
\G{qr} and \g{qw} use the same quorums.
A quorum for the Triangle Quorum Protocol must be a set of nodes who are connected
by the nodes of the set and have at least one node on each of the three
boundaries.
Two valid quorums for the graph in \Rl{fig:trigrid} are \(\{3,4,8,13,14\}\)
and \(\{6,7,8,9,14\}\).
These two quorums show the intersection property of the Triangle Quorum
Protocol.
An optimal quorum for the Triangle Quorum Protocol is a set that only consists
of all nodes of one boundary.

The \g{dsp} was presented in \cite{SKB04:mdq}
\cite{kuag:ggqcrcp:2011, KumarMUGQ:11} it is a \g{qp} that works with \g{lnt}
of arbitrary dimension.
The \g{dsp} is a \g{rfwm} protocol.
Such a \g{lnt} for the 3-dimensional case can be seen in \Rl{fig:dspace}.
Considering the \g{lnt} in \Rl{fig:dspace2} a valid \g{qw} for the \g{dsp}
would be a complete plane in combination with a right-angled line.
Again considering \Rl{fig:dspace2}, a \g{qr} consists of all nodes on a line.
\begin{figure*}[ht]
	\begin{subfigure}[t]{0.48\linewidth}
	\centering
	\includestandalone[width=0.7\textwidth]{twotimestwofigure}
	\caption{A 3-dimensional \g{lnt} used by the D-Space Quorum Protocol}
	\label{fig:dspace}
	\end{subfigure}%
	\begin{subfigure}[t]{0.48\linewidth}
	\centering
	\includestandalone[width=0.7\textwidth]{planeline}
	\caption{A 3-dimensional \g{lnt} used by the D-Space Quorum Protocol}
	\label{fig:dspace2}
	\end{subfigure}
	\caption{Different views on the same \g{lnt} used by \g{dsp}.}
\end{figure*}
The \g{dsp} allows for another way of creating \gp{qw}.
This way can be see in \Rl{fig:dspace}.
Instead of writing a complete plane and a line the quorum can consists of the
nodes of a hyperplane of the \g{lnt} \cite{beutelspacher1998projective}.
A hyperplane is created by compressing the read line axis of the 3-dimensional
space into a 2-dimensional plane.
This joins all nodes which share the same coordinates on the uncompressed
axes.
If the complete plane is then covered by available nodes, viewed from the read
line axis, the write operation is available.
The \g{qp} in \citetitle{kuag:ggqcrcp:2011} and \citetitle{KumarMUGQ:11} based
on the \g{dsp}.

In \cite{RS13:NDS} and \cite{rt2015} the authors presented a quorum
protocols based on \g{tl} that work on arbitrary planar graphs.
The idea behind the \emph{Crossing Protocol} is to create intersecting paths
that completely cross (or: partitions) a given \g{pnt} vertically and/or
horizontally. 
These paths are called \emph{crossings}.
The nodes of the crossings are used as quorums in the Crossing Protocol.
\Gp{qr} have to cross the \g{pnt} vertically or horizontally. 
\Gp{qw} have to cross the \g{pnt} vertically and horizontally. 
Before vertical or horizontal crossing can be defined four boundary sets have
to be defined.
These four sets can be tailored to the given topology and the given
availability requirements.
The sets are called Top (T), Bottom (B), Left (L), and Right (R) set and must
adhere to the following criteria.
\begin{align}
U \in \{T,B,L,R\} &: U \ne \emptyset \label{eq:tblr} \\
\forall v,v' \in U &: \mathbb{V}(\langle v, \dots, v' \rangle) \subset U\\
T \cap L \ne \emptyset &\land \label{eq:tcaplnotempty}
T \cap R \ne \emptyset \\
B \cap L \ne \emptyset &\land
B \cap R \ne \emptyset \\
T \cap B = \emptyset &\land
L \cap R = \emptyset \label{eq:bcaprnotempty}
\end{align}
Consider the graph in \Rl{fig:crossing} let \(B = \{0,1,12\}\), \(T =
\{15,9,3,6\}\), \(L = \{0,2,15\}\), and \(R = \{6,5,8,12\}\) a valid \g{qr}
would be \(\{0,5,13,7,5\}\) and a valid \g{qw} would be
\(\{15,9,3,6,7,11,14,1\}\).
\begin{figure}
\centering
	\includestandalone{crossing}
	\caption{A \g{pnt} useable by the \citetitle{RS13:NDS} and
	\citetitle{rt2015} Quorum Protocol}
	\label{fig:crossing}
\end{figure}
The read and write availability of the Crossing Protocol is solely depended on
the graph structure it is applied to.

\Section{Availability Comparison}
\begin{figure*}[t]
	\centering
	\includestandalone{mapping}
\caption{A graphical representation of the mapping \(M(G,G^{'}) = \{(0,12)\), \((1,8)\), \((2,5)\), 
\((3,6)\), \((4,14)\), \((5,11)\), \((6,7)\), \((7,3)\), \((8,0)\), \((9,2)\), \((10,10)\), 
\((11,9)\), \((12,1)\), \((13,13)\), \((14,4)\), \((15,15)\}\) required by the
\g{tl} to work on the shown \g{pnt}.}
\label{fig:mapping}
\end{figure*}
As discussed earlier, existing \gp{sqp} are usually limited
to a specific \g{lnt}, and are therefore not applicable to \gp{pnt} in the
general case.
In order to usefully and fairly compare \gp{sqp} with each other these
protocols have to analyzed when they are mapped to an identical \g{pnt}.
A generic approach to apply these protocols to any \g{pnt} is to map the
nodes of a protocol's \g{lnt} to the nodes of the \g{pnt}.

None of the presented \gp{qp} have such a mapping feature, nor could such a
mapping algorithm be identified in the literature.
As such an algorithm is mandatory for a useful comparison between any
\gp{sqp}, that do not share a \g{lnt} or \g{pnt}, such an algorithm is
presented below.

\Subsection{The Mapping Approach}
A mapping is a bijection from one graph-structure to another one. 
This requires that the number of nodes in the codomain structure is at least
equal to the number of the nodes of the original structure.
Formally, a mapping \(M(GS,GS')\) from \(GS\) to \(GS'\) is defined as:
\begin{align}
M(GS,GS') &= \{ (v_1,v_1'), \dots (v_n,v_{n}') \} \label{eq:injection1} \\ 
\forall (v,v') \in M &: v \in V, v' \in V'\\
\forall (v,v'),(v,v'') &: v' = v''\\
\forall (v',v),(v'',v) &: v' = v'' \label{eq:injection2}
\end{align}
The (\Rl{eq:injection1}) -- (\Rl{eq:injection2}) define every
node \(v\) of graph \(G\) is uniquely mapped to a node \(v'\) of graph \(G'\). 

In order to apply \gp{sqp} to arbitrary topologies, either
their \g{lnt} must match the \g{pnt} or the nodes of the \g{lnt} have to
be mapped to the \g{pnt}.
As it is highly unlikely that the \g{lnt} matches the \g{pnt}, mapping the
nodes often becomes a necessity.
As an example, consider the \g{lnt} in \Rl{fig:lattice} used by the
\g{tl} and the \g{pnt} presented in \Rl{fig:crossing}.
Obviously, the topologies are different.
Therefore, if the \g{tl} is to be used on the graph in \Rl{fig:crossing} a
mapping is required.
When a mapping is created, the executed \g{sqp} works its \g{pnt}.
For every node selected from the \g{lnt} the mapping selects the mapped node
of the \g{pnt}.
Connections between two or more nodes of the \g{lnt} are recreated with the
nodes of the \g{pnt}.
As mappings are usually not between homomorph graph-structures the length of
recreated connections may differ from the original connection. 
The cost and availability analysis of a mapped protocol therefore has to
consider the nodes required on the \g{pnt} level.
Having mappings that require little additional nodes on the \g{pnt} level in
relation to the \g{lnt} level is therefore a desired goal, as thus the
expected cost and availability characteristics of the \g{sqp} can be traced
the closest.
Consequently, the next question is how to create good or even optimal
mappings.

Before we can find a good or optimal mapping we have to derive a measurement
criteria for mappings.
First we have to reconsider \Rl{eq:avail}. 
We see that the different quorums are valued differently in relation to their
probability of occurrence.
Additionally, some \g{sqp} have special quorums that have higher precedence as
they are cheater to construct and/or have a higher availability.
That means depending on the expected probability of the individual nodes
different parts of a mapping have a different amount of influence on the
resulting availability.
In other words, the ranking of mappings is depended on the assumed
availability of the nodes.

As an example for a hand-optimized mapping from the graph structure \(GS\) in
\Rl{fig:lattice} to the graph structure \(GS^{'}\) in \Rl{fig:crossing}
consider \(M(GS,GS^{'}) = \):
\(\{(0,12),\) \((1,8),\) \((2,5),\) \((3,6),\) \((4,14),\) \((5,11),\)
\((6,7),\) \((7,3),\) \((8,0),\) \((9,2),\) \((10,10),\) \((11,9),\)
\((12,1),\) \((13,13),\) \((14,4),\) \((15,15)\}\).
This mapping can be seen in \Rl{fig:mapping}.
Obviously, such a manual approach is not feasible in the general case.

Considering the displayed mapping allows us to gain a deeper understanding of
the mapping process.
Imagine the \g{tl} yields the diagonal as a \g{qw} consisting of the nodes
\(\{0,5,10,15\}\). 
The nodes of the diagonal get mapped to the nodes \(\{12,11,10,15\}\).
In the \g{lnt} the four nodes where connect by three edges.
In the \g{pnt} none of the four mapped nodes are directly connected.
As mentioned earlier, one assumption of \g{sqp} is that nodes connected in the
\g{lnt} can directly exchange messages between each other.
This communication is required to propagate the data that is to be read and
written.
When mapped, previously directly connected nodes may now require additionally
nodes to be reconnected.
Therefore, additional nodes may be required to propagate messages between
nodes of the selected quorum.
The smallest set of nodes reconnecting the original diagonal \g{qw} in the
\g{pnt} is \(\{15,9,10,13,11,14,12\}\) \footnote{Other reconnecting sets with
seven nodes exists. Choosing them over the selected reconnecting set will not
change the argument.}.
Now it is fairly obvious why the \g{pnt} has to be considered in the
availability analysis.
The availability of the original four node quorum is \(q^{4}\).
The availability of the actual connected seven node quorum is \(q^{7}\).
As \(p \text{ is}\in [0,1)\) the availability of the actual \g{qw} will be
lower than the availability promised by the \g{tl}.

\Subsection{Optimal Mapping}
Below we will present an algorithm that finds the optimal mapping for a
uniform node availability probability range from \(min\) to \(max\) where \(0
\le min \le max < 1\).
Optimal in this case means, the mapping where the integral for a given \g{qp},
between min and max, is the highest.
One part of this algorithm is the reconnecting of nodes of a quorum.
In \Rl{alg:connectnodes} a procedure is given that will find the smallest
subset of nodes of the given \g{pnt} to connected a given set of nodes.
\(\frak{P}(s)\) computes all unique subsets of \(s\), and is known as the
\emph{Power Set} \cite{devlin1979fundamentalsset}.
\begin{algorithm}[h]
\caption{Algorithm to connect nodes.}
\label{alg:connectnodes}
\begin{algorithmic}[1]
\Procedure{AreNodesConnected}{$nodes, GS$}
	\State
	\Return \(\forall n,n' \in nodes : \exists \langle n, \dots, n' \rangle \in GS\)
\EndProcedure
\State
\Procedure{FindSmallest}{$nodesToConnect, GS$}
	\State \# all subsets of nodes including $nodesToConnect$
	\State subsets $\leftarrow$ $\frak{P}(\mathcal{V}(GS) \cap nodes)$
	\State smallest $\leftarrow$ $V(GS)$
	\ForAll{$i \in subsets$}
		\If{AreNodesConnected(\par
			\hskip\algorithmicindent\hskip\algorithmicindent nodesToConnect, GS(i,E(GS))\par
			\hskip\algorithmicindent) $\land$ $|i| \le |smallest|$\par
			\hskip\algorithmicindent}
			\State smallest $\leftarrow$ i
		\EndIf
	\EndFor
	\State
	\Return smallest
\EndProcedure
\end{algorithmic}
\end{algorithm}
The procedure \texttt{FindSmalles} works as follows.
It iterates all possible subsets of \(V(GS)\) which are containing the node to
connect.
For each of these subsets it is tested if their is a path connecting each
element to every other element.
The subset with the fewest number of nodes for which that property is true is
the smallest reconnecting subset of nodes.
\begin{algorithm}[h]
\caption{Algorithm to find optimal mapping.}
\label{alg:optimalmapping}
\begin{algorithmic}[1]
\Procedure{Permutations}{$tup$}
	\State
	\Return all possible permutations of the elements of tuple \(tup\)
\EndProcedure
\State
\Procedure{Func}{$protocol$,$lnt$,$i$,$p$}
	\State result \(\leftarrow\) 0.0
	\State mapping \(\leftarrow\) \(M(lnt,i)\)
	\State pntGS \(\leftarrow\) \(G(lnt,M)\)
	\ForAll{\(q \in protocol(lnt)\)}
		\State \(a\) \(\leftarrow\) \texttt{FindSmallest(q,pntGS)}
		\State \(result \leftarrow result + p^{|a|} * (1 - |a|)^{V(pntGS) - |a|}\)
	\EndFor
	\State
	\Return result
\EndProcedure
\State
\Procedure{OptMapping}{$protocol$,$lnt$,$pnt$,$min$,$max$}
	\State \# \(\mathcal{T}(V)\) is a function that creates a 
	\State \# tuple of the elements of \(V\) such that these element 
	\State \# are ordered by their id.
	\State
	\State \(optMapping \leftarrow\) empty tuple
	\State \(intOptMapping \leftarrow 0.0\)
	\ForAll{\(i \in Permutations(pnt)\)}
		\State \(tmp \leftarrow \int_{min}^{max} Func(protocol,lnt,i,x) dx\)
		\State
		\If{\(tmp > intOptMapping\)}
			\State optMapping \(\leftarrow i\) 
			\State intOptMapping \(\leftarrow tmp\) 
		\EndIf
	\EndFor
	\State
	\Return (optMapping,intOptMapping)
\EndProcedure
\end{algorithmic}
\end{algorithm}
Being able to find the smallest subsets of nodes of a \g{pnt} that reconnects
a quorum created on a \g{lnt}, allows us to determine the optimal mapping from
any replica protocol using any \g{lnt} on every \g{pnt}.
This process is shown in \Rl{alg:optimalmapping}.
The idea behind the procedure \texttt{OptMapping} is to iterate all possible
mappings from the \g{lnt} to the \g{pnt} and then iterate all quorums of the
\g{qp}.
For each mapping, quorum the smallest reconnection set is computed and this is
then used to calculate the availability of the quorum when mapped to the
\g{pnt}.
The next step is compute the integral where the availability \(q\) serves as
the variable of integration.
For practical reasons this step can be approximated by iterating \(q\) in very
small steps (e.g. 0.0.1) between \(min\) and  \(max\).
The mapping with the highest integral value, for a given \g{qp} with a given
node availabilities between \(min,max\), is the optimal mapping.
Comparing the integral values of \gp{qp}, optimally mapped to the same
\g{pnt}, allows for a fair and useful comparison.

\Section{Conclusion}
In this paper we presented and discussed key properties of quorum systems used
for data replication.
Then we used these properties to analyse historically relevant quorum
protocols.
We discussed existing misconceptions and showed remaining difficulties.
In order to correctly compare quorum protocols that use \glsfirst{lnt} and
quorum protocols that use \glsfirst{pnt}, we developed and presented the
algorithms in \Rl{sec:optimalmapping}.

No quorum protocol in more than 30 years has shown to be the right choice
for every given use scenario.

A possible next step in the analysis of quorum protocol is to consider faulty
communication, and to consider nodes that have different availabilities.
This would create analysis that are more realistic and most likely to the
development of new quorum protocols that will try to benefit from the increase
of information.
Unfortunately, these analysis exhibit high complexity as they usually require
to iterate all possible combinations of some specific attribute.
\printbibliography
\end{document}
